\RequirePackage[l2tabu,orthodox]{nag}  % warn about common LaTeX pitfalls
\RequirePackage[ascii]{inputenc}  % input is 7-bit ASCII
\RequirePackage{fixltx2e}  % fix LaTeX2e kernel bugs

\documentclass[11pt,twoside]{article}
\usepackage{color}
\usepackage{graphicx}
\graphicspath{ {image/} }
\usepackage{calc}  % arithmetic in length parameters
\usepackage{enumitem}  % more control over list formatting
\usepackage{fancyhdr}  % simpler headers and footers
\usepackage[margin=1in]{geometry}  % page layout
\usepackage{lastpage}  % for last page number
\usepackage{relsize}  % easier font size changes
\usepackage[normalem]{ulem}  % smarter underlining
\usepackage{url}  % verb-like typesetting of URLs
\usepackage{xfrac}  % nicer looking simple fractions for text and math
\usepackage{longtable}
\usepackage{tikz}
\usepackage{array}
\usepackage{tikz-timing}
\usetikzlibrary{arrows, shapes, backgrounds,fit}
\usepackage{tkz-graph}
% Set up fonts.
\usepackage[T1]{fontenc}  % use true 8-bit fonts
\usepackage{slantsc}  % allow slanted small-caps
\usepackage{microtype}  % perform various font optimizations
% Use Palatino-based monospace instead of kpfonts' default.
%\usepackage{newpxtext}
\ttfamily
\DeclareFontShape{T1}{\ttdefault}{m}{scsl}{<->ssub*\ttdefault/m/sc}{}
\DeclareFontShape{T1}{\ttdefault}{b}{scsl}{<->ssub*\ttdefault/b/sc}{}
% "Kepler" fonts.
\usepackage[nott,notextcomp]{kpfonts}
% Use curvier Latin Modern brackets instead of kpfonts' glyphs.
\DeclareSymbolFont{lmsymb}     {OMS}{lmsy}{m}{n}
\DeclareSymbolFont{lmlargesymb}{OMX}{lmex}{m}{n}
\DeclareMathDelimiter{\rbrace}{\mathclose}{lmsymb}{"67}{lmlargesymb}{"09}
\DeclareMathDelimiter{\lbrace}{\mathopen}{lmsymb}{"66}{lmlargesymb}{"08}

% Page layout: stretch text to fill up page.
\addtolength\footskip{.25\headheight}
\flushbottom

% Common list settings.

% Common macros.
\input{macros}
\newcommand*\st{\mathrel{|}}  % "such that" for set extension

% Headings.
\pagestyle{fancy}
\let\headrule\empty
\let\footrule\empty
\lhead{CSC\,336\,H1}
\chead{\large\scshape Assignment \#\,3}
\rhead{\scshape Fall 2015}
\lfoot{\scshape Dept.\@ of Computer Science, University of Toronto,
       St.~George Campus}
\cfoot{}
\rfoot{\scshape page \thepage\space of \pageref{LastPage}}


\begin{document}

\begin{enumerate}[leftmargin=0pt]
% question 1
\item
	\begin{enumerate}
	% part a
	\item  
	\[|g_1'(x)| = |\frac{2x}{3}| \Rightarrow |g_1'(2)| = \frac{4}{3} > 1 \]
	\[|g_2'(x)| = |\frac{3}{2 \sqrt{3x-2}}|  \Rightarrow |g_2'(2)| = \frac{3}{4} < 1\]
	\[|g_3'(x)| = |\frac{2}{x^2}|  \Rightarrow |g_3'(2)| = \frac{1}{2} < 1\]
	\[|g_4'(x)| = |\frac{2x(2x-3) - 2(x^2-2)}{(2x-3)^2}| \Rightarrow |g_4'(2)| = \frac{4-4}{1} = 0\]
	Then we know that $ |g_1'(2)|$  is bigger than 1, which means divergence. $|g_2'(2) = \frac{3}{4}|$ means it is 
	linear convergence with constant  $|\frac{3}{4}|$.   $|g_3'(2) = \frac{1}{2}|$ means it is linear convergence with constant  $|\frac{1}{2}|$. And  $|g_4'(2) = 0|$ means it is quadratic convergence.
 	%part b
	\item
	\[ \includegraphics[scale=0.5]{1b} \]
	\[ \includegraphics[scale=0.5]{1br} \]
	Clearly, the converging rate is approximately like what we calculated. 
	\end{enumerate}
% question 2
\item
	\begin{enumerate}
	\item Starting with the secant method update formula is given as
				\[x_{k+1} = x_k - \frac{f(x_k)(x_k-x_{k-1})}{f(x_k)-f(x_{k-1})}\]
				\[=  \frac{x_k(f(x_k)-f(x_{k-1})) - f(x_k)(x_k-x_{k-1})}{f(x_k)-f(x_{k-1})}\]
				\[=  \frac{x_k f(x_k) - x_k f(x_{k-1}) - x_kf(x_k)+x_{k-1}f(x_k)}{f(x_k)-f(x_{k-1})}\]
				\[=  \frac{x_{k-1} f(x_k) - x_k f(x_{k-1})} {f(x_k)-f(x_{k-1})}\]
	\item  When we close to solution then $x_{k-1}$ and $x_k$ are close to each other which mean there difference is close to $0$, and also can cause  catastrophic cancellation. For formula in $part (a) $ it is hard for us to get rid of cancellation. For formula in $part (b)$, catastrophic cancellation is the only thing affecting $x_{k+1}$

	\end{enumerate}

% question 3
\item
	\[ \includegraphics[scale=0.35]{3b} \]
	\[ \includegraphics[scale=0.35]{3n} \]
	\[ \includegraphics[scale=0.35]{3s} \]
	For termination criteria, 
	\begin{itemize}[label = {}]
		\item using $b-a > tol$ for bisection;
		\item using $f(x) > tol $ for Newton's method;
		\item using $f(x) > tol $ for secant.
	\end{itemize}
	\begin{enumerate}
	% part a
	\item 	\[f(x): x^3 - 2x - 5 = 0\]
		  	\[f'(x): 3x^2 -2\]
		% bisection
		$bisection: $
		 \begin{longtable}{|>{\tiny}p{0.5in}|>{\tiny} p{0.5in}| >{\tiny}p{0.5in}|>{\tiny}p{0.5in}|	
		 	>{\tiny} p{0.8in}|}\hline
			iteration&interval&tolerance&root&convergence \ rate \\[0.1in]\hline
			12 &[1,3]&0.001&2.0947&linear around(0.5)\\[0.1in] \hline
		\end{longtable} 
		%newton
		$Newton's method:$
		 \begin{longtable}{|>{\tiny}p{0.5in}|>{\tiny} p{0.5in}| >{\tiny}p{0.5in}|>{\tiny}p{0.5in}|	
		 	>{\tiny} p{0.8in}|}\hline
			iteration&initial \ guess&tolerance&root&convergence \ rate \\[0.1in]\hline
			8 &1&0.001&2.0946&linear around $10^{-1}$\\[0.1in] \hline
		\end{longtable} 
		%secant
		$secant:$
		\begin{longtable}{|>{\tiny}p{0.5in}|>{\tiny} p{0.5in}| >{\tiny}p{0.5in}|>{\tiny}p{0.5in}|	
		 	>{\tiny} p{0.8in}|}\hline
			iteration&initial \ guess&tolerance&root&convergence \ rate \\[0.1in]\hline
			8 &1,3&0.001&2.0946& linear($10^{-1}$)\\[0.1in] \hline
		\end{longtable} 
		$fzero:$ \\
		Using matlab  $fzero$ with initial guess $1$, we get a root $2.0946$.

	% part b
	\item 	\[ e^{-x} = x\]
			\[-e^{-x} = 1\]
		% bisection
		$bisection: $
		 \begin{longtable}{|>{\tiny}p{0.5in}|>{\tiny} p{0.5in}| >{\tiny}p{0.5in}|>{\tiny}p{0.5in}|	
		 	>{\tiny} p{0.8in}|}\hline
			iteration&interval&tolerance&root&convergence \ rate \\[0.1in]\hline
			11 &[0,1]&0.001&0.5674&linear around(0.5)\\[0.1in] \hline
		\end{longtable} 
		%newton
		$Newton's method:$
		 \begin{longtable}{|>{\tiny}p{0.5in}|>{\tiny} p{0.5in}| >{\tiny}p{0.5in}|>{\tiny}p{0.5in}|	
		 	>{\tiny} p{0.8in}|}\hline
			iteration&initial \ guess&tolerance&root&convergence \ rate \\[0.1in]\hline
			4 &0&0.001&0.5671&linear($10^{-1}$)\\[0.1in] \hline
		\end{longtable} 
		%secant
		$secant:$
		\begin{longtable}{|>{\tiny}p{0.5in}|>{\tiny} p{0.5in}| >{\tiny}p{0.5in}|>{\tiny}p{0.5in}|	
		 	>{\tiny} p{0.8in}|}\hline
			iteration&initial \ guess&tolerance&root&convergence \ rate \\[0.1in]\hline
			5 &0,1&0.001&0.5672& linear($10^{-1}$)\\[0.1in] \hline
		\end{longtable} 
		$fzero:$ \\
		Using matlab  $fzero$ with initial guess $1$, we get a root $0.5671$.

	%part c
	\item \[x\sin(x) = 1\]
		 \[\sin(x) + x\cos(x) = 0\]
		% bisection
		$bisection: $
		 \begin{longtable}{|>{\tiny}p{0.5in}|>{\tiny} p{0.5in}| >{\tiny}p{0.5in}|>{\tiny}p{0.5in}|	
		 	>{\tiny} p{0.8in}|}\hline
			iteration&interval&tolerance&root&convergence \ rate \\[0.1in]\hline
			11 &[1,2]&0.001&1.1143&linear around(0.5)\\[0.1in] \hline
		\end{longtable} 
		%newton
		$Newton's method:$
		 \begin{longtable}{|>{\tiny}p{0.5in}|>{\tiny} p{0.5in}| >{\tiny}p{0.5in}|>{\tiny}p{0.5in}|	
		 	>{\tiny} p{0.8in}|}\hline
			iteration&initial \ guess&tolerance&root&convergence \ rate \\[0.1in]\hline
			2 &0.5&0.001&1.1147&linear($10^{-1}$)\\[0.1in] \hline
		\end{longtable} 
		%secant
		$secant:$
		\begin{longtable}{|>{\tiny}p{0.5in}|>{\tiny} p{0.5in}| >{\tiny}p{0.5in}|>{\tiny}p{0.5in}|	
		 	>{\tiny} p{0.8in}|}\hline
			iteration&initial \ guess&tolerance&root&convergence \ rate \\[0.1in]\hline
			5 &1,2&0.001&1.1142&super linear (1.5)\\[0.1in] \hline
		\end{longtable} 
		$fzero:$ \\
		Using matlab  $fzero$ with initial guess $1$, we get a root $1.1142$.
	%part d
	\item \[x^3 -3x^2 +3x -1 = 0\]
		 \[3x^2 -6x + 3 = 0\]
		% bisection
		$bisection: $
		 \begin{longtable}{|>{\tiny}p{0.5in}|>{\tiny} p{0.5in}| >{\tiny}p{0.5in}|>{\tiny}p{0.5in}|	
		 	>{\tiny} p{0.8in}|}\hline
			iteration&interval&tolerance&root&convergence \ rate \\[0.1in]\hline
			12 &[0,3]&0.001&0.9998& linear(0.5)\\[0.1in] \hline
		\end{longtable} 
		%newton
		$Newton's method:$
		 \begin{longtable}{|>{\tiny}p{0.5in}|>{\tiny} p{0.5in}| >{\tiny}p{0.5in}|>{\tiny}p{0.5in}|	
		 	>{\tiny} p{0.8in}|}\hline
			iteration&initial \ guess&tolerance&root&convergence \ rate \\[0.1in]\hline
			7 &0&0.001&0.9122&linear(0.66)\\[0.1in] \hline
		\end{longtable} 
		%secant
		$secant:$
		\begin{longtable}{|>{\tiny}p{0.5in}|>{\tiny} p{0.5in}| >{\tiny}p{0.5in}|>{\tiny}p{0.5in}|	
		 	>{\tiny} p{0.8in}|}\hline
			iteration&initial \ guess&tolerance&root&convergence \ rate \\[0.1in]\hline
			11 &0,3&0.001&0.9232& linear(0.75)\\[0.1in] \hline
		\end{longtable} 
		$fzero:$ \\
		Using matlab  $fzero$ with initial guess $0$, we get a root $1.0000$.
	\end{enumerate}
% question 4
\item
	\begin{enumerate}
	% part a
	\item 
	\[ f(x) =\sqrt[3]{1-\frac{3}{4x}} = 0 \Rightarrow  1-\frac{3}{4x} = 0\]
	\[4x = 3 \Rightarrow x =\frac{3}{4}\]
	Hence, clearly function $ f(x) =\sqrt[3]{1-\frac{3}{4x}}$ only has one root which is $\frac{3}{4}$.
	%part b
	
	\item
	\[ \includegraphics[scale=0.5]{4b} \]
	 5 plots of $Newton's \ method$ with random starting points show below, and the red line indicates the last iteration of the total 50 iterations.
		\[ \includegraphics[scale=0.35]{plot(16-8175).png} \]
		\[ \includegraphics[scale=0.35]{plot1(2-4371).png} \]
		\[ \includegraphics[scale=0.35]{plot(1-3285).png} \]
		\[ \includegraphics[scale=0.35]{plot(0-5184).png} \]
		\[ \includegraphics[scale=0.35]{plot2(0-7422).png} \]
	% part c
	\item From the plots, we could observe,
		\begin{itemize}
		\item Base on different initial guess with fixed iteration the $Newton's \ method$ might converge or diverge.
		\item For a $Newton's \ method$ with some initial guess, the function may converge for some certain iterations (i.e it may first converge then diverge). 
		\end{itemize}
	\end{enumerate}
	
\end{enumerate}

\end{document}